%!TEX root = edance.tex
%%%%%%%%%%%%%%%%
%  APPENDIX G  %
%%%%%%%%%%%%%%%%
\chapter{Appendix: Glossary of Terms}
%\graphicspath{{./figs_opamp_real/}}
\label{app:glossary}
%%%%%%%%%%%%%%%%%%%%%%%%%%%%%%%%%%%%%%%%%%%%%%%%%%%%%%%%%%%%%%%%%%%%%%%%%%%%%%%%%%%%%%
%                                   SECTION A                                        %
%%%%%%%%%%%%%%%%%%%%%%%%%%%%%%%%%%%%%%%%%%%%%%%%%%%%%%%%%%%%%%%%%%%%%%%%%%%%%%%%%%%%%%
\section{A}
    \textbf{Acceptor} - An element used in doping (Group $III$ for $Si$) that has less electrons than the semiconductor it is being injected into in order to accept an electron from the valence band in order to free a hole to be available for conducting current; increases the hole concentration.
    %%%%%%%%%%%%%%%%%
    \vspace{0.15cm}
    %%%%%%%%%%%%%%%%%
    \begin{itemize}
        \setlength\itemsep{0.5em}
        \item{If the energy absorbed at the acceptor is precisely equal to the electron binding energy, the released hole will have the lowest possible energy in the valence band, namely $E_V$.}
    \end{itemize}
%%%%%%%%%%%%%%%%%
\vspace{0.5cm}
%%%%%%%%%%%%%%%%%
    \textbf{Acoustic Phonon} - Coherent movements of atoms of the lattice out of their equilibrium positions.
    %%%%%%%%%%%%%%%%%
    \vspace{0.15cm}
    %%%%%%%%%%%%%%%%%
    \begin{itemize}
        \setlength\itemsep{0.5em}
        \item{Occur at lower temperatures and energy.}
    \end{itemize}
%%%%%%%%%%%%%%%%%
\vspace{0.5cm}
%%%%%%%%%%%%%%%%%
    \textbf{Amphoteric Dopant} - An impurity that can act as either a donor or acceptor.
    %%%%%%%%%%%%%%%%%
    \vspace{0.15cm}
    %%%%%%%%%%%%%%%%%
    \begin{itemize}
        \setlength\itemsep{0.5em}
        \item{$Si$ typically replaces $Ga$ in the $GaAs$ ($III-V$) lattice, and is a popular $N$\emph{-type} dopant.}
        \item{The size of the dopant ($Si$) compared to the $III-V$ elements determines which species will be replaced, and whether the dopant becomes a donor or an acceptor.}
    \end{itemize}
%%%%%%%%%%%%%%%%%
\vspace{0.5cm}
%%%%%%%%%%%%%%%%%
    \textbf{Anode} - In an electrical device, the anode is the \textit{positive} terminal in which conventional current enters.
    
%%%%%%%%%%%%%%%%%
\vspace{0.5cm}
%%%%%%%%%%%%%%%%%
    \textbf{Annealing} - A heat treatment that alters the physical and sometimes chemical properties of a material to increase its ductility and reduce its hardness, making it more workable.
    %%%%%%%%%%%%%%%%%
    \vspace{0.15cm}
    %%%%%%%%%%%%%%%%%
    \begin{itemize}
        \setlength\itemsep{0.5em}
        \item{It involves heating a material above its recrystallization temperature, maintaining a suitable temperature for an appropriate amount of time and then cooling.}
    \end{itemize}
%%%%%%%%%%%%%%%%%%%%%%%%%%%%%%%%%%%%%%%%%%%%%%%%%%%%%%%%%%%%%%%%%%%%%%%%%%%%%%%%%%%%%%
%                                   SECTION B                                        %
%%%%%%%%%%%%%%%%%%%%%%%%%%%%%%%%%%%%%%%%%%%%%%%%%%%%%%%%%%%%%%%%%%%%%%%%%%%%%%%%%%%%%%
\section{B}
    \textbf{Band Bending} - The resulting variation of $E_C$ and $E_V$ due to an electric field.
    %%%%%%%%%%%%%%%%%
    \vspace{0.15cm}
    %%%%%%%%%%%%%%%%%
    \begin{itemize}
        \setlength\itemsep{0.5em}
        \item{null}
    \end{itemize}
%%%%%%%%%%%%%%%%%
\vspace{0.5cm}
%%%%%%%%%%%%%%%%%
    \textbf{Band Gap} - The intervening energy gap between the upper band and lower band of allowed energy states.  In $Si$, this value is typically around $1.12\;eV$.
    %%%%%%%%%%%%%%%%%
    \vspace{0.15cm}
    %%%%%%%%%%%%%%%%%
    \begin{itemize}
        \setlength\itemsep{0.5em}
        \item{null}
    \end{itemize}
%%%%%%%%%%%%%%%%%
\vspace{0.5cm}
%%%%%%%%%%%%%%%%%
    \textbf{Biasing} - The setting of initial operating conditions (current and voltage) of an active device in an amplifier.
    %%%%%%%%%%%%%%%%%
    \vspace{0.15cm}
    %%%%%%%%%%%%%%%%%
    \begin{itemize}
        \setlength\itemsep{0.5em}
        \item{null}
    \end{itemize}
%%%%%%%%%%%%%%%%%
\vspace{0.5cm}
%%%%%%%%%%%%%%%%%
    \textbf{Bipolar Transistor} - A type of transistor that uses both electrons and holes as charge carriers.  It has three different sections of semiconductor material known as the base, collector, and emitter.
    %%%%%%%%%%%%%%%%%
    \vspace{0.15cm}
    %%%%%%%%%%%%%%%%%
    \begin{itemize}
        \setlength\itemsep{0.5em}
        \item{A bipolar transistor allows a small current injected at one of its terminals to control a much larger current flowing between the terminals, making the device capable of amplification or switching.}
        \item{$NPN$ BJT's have a $P$\emph{-type} base, and an $N$\emph{-type} collector.  The emitter region is heavily doped $N^+$.}
        \item{$PNP$ BJT's have an $N$\emph{-type} base, and a $P$\emph{-type} collector.  The emitter region is heavily doped $P^+$.}
        \item{$NPN$ transistors exhibit higher transconductance and speed than $PNP$ transistors, because the electron mobility is larger than the hole mobility.}
        \item{The collector current, $I_C$, is determined by the rate of electron injection from the emitter into the base.}
    \end{itemize}
%%%%%%%%%%%%%%%%%
\vspace{0.5cm}
%%%%%%%%%%%%%%%%%
    \textbf{Boltzmann Distribution} - A probability distribution or probability measure that gives the probability that a system will be in a certain state as a function of that state's energy and the temperature of the system. 
%%%%%%%%%%%%%%%%%%%%%%%%%%%%%%%%%%%%%%%%%%%%%%%%%%%%%%%%%%%%%%%%%%%%%%%%%%%%%%%%%%%%%%
%                                   SECTION C                                        %
%%%%%%%%%%%%%%%%%%%%%%%%%%%%%%%%%%%%%%%%%%%%%%%%%%%%%%%%%%%%%%%%%%%%%%%%%%%%%%%%%%%%%%
\section{C}
    \textbf{Carriers} - The entities that transport charge from place to place inside a material, and give rise to electrical currents.  Most of the carriers are grouped energetically in the near vicinity of the conduction band edge or valence band edge.

%%%%%%%%%%%%%%%%%
\vspace{0.5cm}
%%%%%%%%%%%%%%%%%
    \textbf{Cathode} - In an electrical device, the cathode is the \textit{negative} terminal in which conventional current exits.
    
%%%%%%%%%%%%%%%%%
\vspace{0.5cm}
%%%%%%%%%%%%%%%%%
\noindent
    \textbf{Charge Neutrality} - A relationship that provides the general tie between the carrier and dopant concentrations.  It says that a \emph{uniformly doped} semiconductor must have no net charge.
    %%%%%%%%%%%%%%%%%
    \vspace{0.15cm}
    %%%%%%%%%%%%%%%%%
    \begin{itemize}
        \setlength\itemsep{0.5em}
        \item Below is the derivation for charge neutrality.  Remember that donors become positively charged ions, and acceptors become negatively charged ions.
        {\begin{align*}
            0 &= p + n + donors + acceptors\\
            &= qp - qn + N_D - N_A\\
            &\implies \boxed{n + N_A = p + N_D}
        \end{align*}}
    \end{itemize}
%%%%%%%%%%%%%%%%%
\vspace{0.5cm}
%%%%%%%%%%%%%%%%%
    \textbf{Compensated Semiconductor} - When $N_A$ and $N_D$ are comparable and nonzero, the material is said to be compensated, in which case $N_A$ and $N_D$ must be retained in all carrier concentration expressions.

%%%%%%%%%%%%%%%%%
\vspace{0.5cm}
%%%%%%%%%%%%%%%%%
\noindent
    \textbf{Conduction Band} - The upper band of allowed energy states in $Si$.

%%%%%%%%%%%%%%%%%
\vspace{0.5cm}
%%%%%%%%%%%%%%%%%
\noindent
    \textbf{Conductivity} -

%%%%%%%%%%%%%%%%%
\vspace{0.5cm}
%%%%%%%%%%%%%%%%%
\noindent
    \textbf{Current} - The charge per unit time crossing an arbitrarily chosen plane of observation oriented normal to the direction of current flow.
%%%%%%%%%%%%%%%%%%%%%%%%%%%%%%%%%%%%%%%%%%%%%%%%%%%%%%%%%%%%%%%%%%%%%%%%%%%%%%%%%%%%%%
%                                   SECTION D                                        %
%%%%%%%%%%%%%%%%%%%%%%%%%%%%%%%%%%%%%%%%%%%%%%%%%%%%%%%%%%%%%%%%%%%%%%%%%%%%%%%%%%%%%%
\section{D}
    \textbf{Degenerate Semiconductor} - A semiconductor with such a high level of doping that the material starts to act more like a metal than as a semiconductor. Unlike non-degenerate semiconductors, these kind of semiconductor do not obey law of mass action, which relates intrinsic carrier concentration with temperature and band gap.
    %%%%%%%%%%%%%%%%%
    \vspace{0.15cm}
    %%%%%%%%%%%%%%%%%
    \begin{itemize}
        \setlength\itemsep{0.5em}
        \item{Whenever $E_F$ lies in the band gap closer than $3kT$ to either band edge, or actually penetrates one of the bands.}
    \end{itemize}
%%%%%%%%%%%%%%%%%
\vspace{0.5cm}
%%%%%%%%%%%%%%%%%
    \textbf{Density of States} - The energy distribution of allowed states; tells us how many states exist at a given energy.
    %%%%%%%%%%%%%%%%%
    \vspace{0.15cm}
    %%%%%%%%%%%%%%%%%
    \begin{itemize}
        \setlength\itemsep{0.5em}
        \item{For $Si$ and $Ge$ the density of states are of the same order of magnitude at corresponding energies.}
    \end{itemize}
%%%%%%%%%%%%%%%%%
\vspace{0.5cm}
%%%%%%%%%%%%%%%%%
    \textbf{Diffusion} - A process whereby particles tend to spread out or redistribute as a result of their random thermal motion, migrating on a macroscopic scale from regions of high particle concentration into regions of low particle concentration.  Thermal motion, not interparticle repulsion, is the enabling action behind the diffusion process.

%%%%%%%%%%%%%%%%%
\vspace{0.5cm}
%%%%%%%%%%%%%%%%%
\noindent
    \textbf{Direct Bandgap} - The minimum energy state in the conduction band and the maximum energy state in the valence band occur at the same wave vector, $k$.  This allows carriers to directly emit a photon, or recombine.
    %%%%%%%%%%%%%%%%%
    \vspace{0.15cm}
    %%%%%%%%%%%%%%%%%
    \begin{itemize}
        \setlength\itemsep{0.5em}
        \item{Dominant R-G mechanism is band-to-band.}
    \end{itemize}
%%%%%%%%%%%%%%%%%
\vspace{0.5cm}
%%%%%%%%%%%%%%%%%
    \textbf{Donor} - An element used in doping (Group $V$ for $Si$) that has more electrons than the semiconductor it is being injected into in order to donate an electron to be available for conducting current; increases the electron concentration.
    %%%%%%%%%%%%%%%%%
    \vspace{0.15cm}
    %%%%%%%%%%%%%%%%%
    \begin{itemize}
        \setlength\itemsep{0.5em}
        \item{If the energy absorbed at the donor is precisely equal to the electron binding energy, the released electron will have the lowest possible energy in the conducting band, namely $E_C$.}
    \end{itemize}
%%%%%%%%%%%%%%%%%
\vspace{0.5cm}
%%%%%%%%%%%%%%%%%
    \textbf{Doping} - The addition of controlled amounts of specific impurity atoms with the express purpose of increasing either the electron or the hole concentration.
    %%%%%%%%%%%%%%%%%
    \vspace{0.15cm}
    %%%%%%%%%%%%%%%%%
    \begin{itemize}
        \setlength\itemsep{0.5em}
        \item{All doped semiconductors become intrinsic at sufficiently high temperature where $n_i \gg \lvert N_D - N_A \rvert$.}
    \end{itemize}
%%%%%%%%%%%%%%%%%
\vspace{0.5cm}
%%%%%%%%%%%%%%%%%
    \textbf{Drift} - Charged-particle motion in response to an applied electric-field.
    %%%%%%%%%%%%%%%%%
    \vspace{0.15cm}
    %%%%%%%%%%%%%%%%%
    \begin{itemize}
        \setlength\itemsep{0.5em}
        \item{Because of collisions with ionized impurity atoms and thermally agitated lattice atoms, carrier acceleration is frequently interrupted and said to be scattered.}
        \item{Averaging over all electrons or holes at any given time, the resultant motion of each carrier type can be described in terms of a contant drift velocity, $\vec{v_d}$.}
        \item{Electrons in the conduction band and holes in the valence band gain and lose energy via collisions with the semiconductor lattice and are nowhere near stationary even under equilibrium conditions.}
    \end{itemize}
%%%%%%%%%%%%%%%%%
\vspace{0.5cm}
%%%%%%%%%%%%%%%%%
    \textbf{Drift Current} - When an electric field is applied across a semiconductor, the resulting force on the carriers tends to accelerate the $+q$ charged holes in the direction of the electric field, and the $-q$ charged electrons in the direction opposite the electric field.
%%%%%%%%%%%%%%%%%%%%%%%%%%%%%%%%%%%%%%%%%%%%%%%%%%%%%%%%%%%%%%%%%%%%%%%%%%%%%%%%%%%%%%
%                                   SECTION E                                        %
%%%%%%%%%%%%%%%%%%%%%%%%%%%%%%%%%%%%%%%%%%%%%%%%%%%%%%%%%%%%%%%%%%%%%%%%%%%%%%%%%%%%%%
\section{E}
    \textbf{Early Effect} - the variation in the effective width of the base in a bipolar junction transistor (BJT) due to a variation in the applied base-to-collector voltage.
    %%%%%%%%%%%%%%%%%
    \vspace{0.15cm}
    %%%%%%%%%%%%%%%%%
    \begin{itemize}
        \setlength\itemsep{0.5em}
        \item{A greater reverse bias across the collector–base junction, for example, increases the collector–base depletion width, thereby decreasing the width of the charge carrier portion of the base.}
    \end{itemize}
%%%%%%%%%%%%%%%%%
\vspace{0.5cm}
%%%%%%%%%%%%%%%%%
    \textbf{Eddy Current} - Loops of electrical current induced within conductors by a changing magnetic field in the conductor according to Faraday's law of induction.

%%%%%%%%%%%%%%%%%
\vspace{0.5cm}
%%%%%%%%%%%%%%%%%
\noindent
    \textbf{Effective Mass} - The apparent mass of electrons and holes inside of the crystal structure of a semiconductor, which is different from the mass of electrons within a vacuum due to the atomic structure of the semiconductor crystal, and how the atoms interact with each other inside of it.
    %%%%%%%%%%%%%%%%%
    \vspace{0.15cm}
    %%%%%%%%%%%%%%%%%
    \begin{itemize}
        \setlength\itemsep{0.5em}
        \item{Inside the crystal electrons will collide with the semiconductor atoms, which causes a periodic negative acceleration of the carriers. In addition to this, electrons inside of the semiconductor crystal are also subject to complex crystalline fields in addition to the electric field.}
        \item{The motion of carriers inside a semiconductor crystal is a quantum mechanical phenomenon, because the electrons moving through a solid experience a specific potential energy which dictates the mass of a wave associated with the electron.  This is why assigning an effective mass allows us to treat electrons and holes as quasi-classical particles and use classical particle relationships in most analyses.}
        \item{The effective mass is a result of electron interaction with the lattice (phonons).  Holes spend more time interacting with phonons because they have a smaller velocity than electrons.  This results in holes usually having a larger effective mass.}
        \item{The effective mass depends on the effective potential the electron or hole feels when moving through the lattice.  The potential seen by the electrons and the holes is different, and this is why they have different effective masses.}
    \end{itemize}
%%%%%%%%%%%%%%%%%
\vspace{0.5cm}
%%%%%%%%%%%%%%%%%
    \textbf{Einstein Relation} - For a particle with electrical charge $q$, its electrical mobility $\mu_q$ is related to its generalized mobility $\mu$ by the equation $\mu = \frac{\mu_q}{q}$. The parameter $\mu_q$ is the ratio of the particle's terminal drift velocity to an applied electric field. Hence, the equation in the case of a charged particle is given as:
    \begin{equation}
        \large{D = \frac{\mu_q\;k_B\;T}{q},}
    \end{equation}
    where,
    %%%%%%%%%%%%%%%%%
    \vspace{0.15cm}
    %%%%%%%%%%%%%%%%%
    \begin{itemize}
        \setlength\itemsep{0.5em}
        \item{$D$ is the diffusion coefficient ($m^2s^{-1}$).}
        \item{$\mu_q$ is the electrical mobility ($m^2V^-1s^{-1}$).}
        \item{$q$ is the electrical charge of a particle in coulombs ($C$).}
        \item{$T$ is the electron temperature or ion temperature in plasma ($K$).}
    \end{itemize}
    %%%%%%%%%%%%%%%%%
    \vspace{0.15cm}
    %%%%%%%%%%%%%%%%%
    \begin{itemize}
        \setlength\itemsep{0.5em}
        \item{The Einstein relation is derived in equilibrium, however it is valid in non-equilibrium.}
    \end{itemize}
%%%%%%%%%%%%%%%%%
\vspace{0.5cm}
%%%%%%%%%%%%%%%%%
    \textbf{Electrochemical Potential} -In electrochemistry, the electrochemical potential is a thermodynamic measure of chemical potential that does not omit the energy contribution of electrostatics.
    %%%%%%%%%%%%%%%%%
    \vspace{0.15cm}
    %%%%%%%%%%%%%%%%%
    \begin{itemize}
        \setlength\itemsep{0.5em}
        \item{Electrochemical potential is expressed in the unit of $J/mol$.}
        \item{The difference in Fermi energy on either end of a device.}
    \end{itemize}
%%%%%%%%%%%%%%%%%
\vspace{0.5cm}
%%%%%%%%%%%%%%%%%
    \textbf{Energy Bands} - By conceptually bringing $N$ silicon atoms closer and closer together, the interatomic forces lead to a spread in the allowed energies which give rise to closely spaced sets of allowed states known as energy bands.  At the interatomic distance corresponding to the $Si$ lattice spacing , the distribution of allowed states consists of two bands separated by an intervening energy gap.
    %%%%%%%%%%%%%%%%%
    \vspace{0.15cm}
    %%%%%%%%%%%%%%%%%
    \begin{itemize}
        \setlength\itemsep{0.5em}
        \item{For every possible momentum state there is another state with an oppositely directed momentum of equal magnitude.}
        \item{Thus, if a band is completely filled with electrons, the \emph{net} momentum of the electrons in the band is always identically zero.}
        \item{No current can arise from the electrons in a completely filled energy band.}
        \item{When an electric field exists inside a material, the band energies become a function of position.}
    \end{itemize}
%%%%%%%%%%%%%%%%%
\vspace{0.5cm}
%%%%%%%%%%%%%%%%%
    \textbf{Equilibrium} - A term used to describe the unperturbed state of a system.  For a semiconductor in equilibrium conditions this means that there are no external voltages, magnetic fields, stresses, or other perturbing forces acting on the semiconductor.
    %%%%%%%%%%%%%%%%%
    \vspace{0.15cm}
    %%%%%%%%%%%%%%%%%
    \begin{itemize}
        \setlength\itemsep{0.5em}
        \item{All observables are invariant with time.}
        \item{Under equilibrium conditions the Fermi level inside a material or a group of materials in intimate contact is invariant as a function of position.}
        \item{Under equilibrium conditions the total current is zero, and the drift and diffusion components of a given carrier are required to be of equal magnitude, but opposite polarity.}
        \item{Even under equilibrium conditions, nonuniform doping will give rise to carrier concentration gradients, a built-in electric field, and non-zero current components.}
    \end{itemize}
%%%%%%%%%%%%%%%%%
\vspace{0.5cm}
%%%%%%%%%%%%%%%%%
    \textbf{Equipotential} - A region in space where every point in it is at the same potential.

%%%%%%%%%%%%%%%%%
\vspace{0.5cm}
%%%%%%%%%%%%%%%%%
\noindent
    \textbf{Extrinsic Semiconductor} - A doped semiconductor; a semiconductor whose properties are controlled by added impurity atoms.
%%%%%%%%%%%%%%%%%%%%%%%%%%%%%%%%%%%%%%%%%%%%%%%%%%%%%%%%%%%%%%%%%%%%%%%%%%%%%%%%%%%%%%
%                                   SECTION F                                        %
%%%%%%%%%%%%%%%%%%%%%%%%%%%%%%%%%%%%%%%%%%%%%%%%%%%%%%%%%%%%%%%%%%%%%%%%%%%%%%%%%%%%%%
\section{F}
    \textbf{Fermi Energy} - A concept in quantum mechanics usually referring to the energy difference between the highest and lowest occupied single-particle states in a quantum system of non-interacting fermions at absolute zero temperature.
    The term "Fermi energy" is often used to refer to a different yet closely related concept, the "Fermi level".  There are a few key differences between the Fermi level and Fermi energy:
    %%%%%%%%%%%%%%%%%
    \vspace{0.15cm}
    %%%%%%%%%%%%%%%%%
    \begin{itemize}
        \setlength\itemsep{0.5em}
        \item{The Fermi energy is only defined at absolute zero, while the Fermi level is defined for any temperature.}
        \item{The Fermi energy is an energy difference (usually corresponding to a kinetic energy), whereas the Fermi level is a total energy level including kinetic energy and potential energy.}
        \item{The Fermi energy can only be defined for non-interacting fermions (where the potential energy or band edge is a static, well defined quantity), whereas the Fermi level remains well defined even in complex interacting systems, at thermodynamic equilibrium.}
    \end{itemize}
%%%%%%%%%%%%%%%%%
\vspace{0.5cm}
%%%%%%%%%%%%%%%%%
    \textbf{Fermi Function} - Specifies how many of the existing states at a given energy $E$ will be filled with an electron, or equivalently, specifies, under equilibrium conditions, the probability that an available state at an energy $E$ will be occupied by an electron.
    %%%%%%%%%%%%%%%%%
    \vspace{0.15cm}
    %%%%%%%%%%%%%%%%%
    \begin{itemize}
        \setlength\itemsep{0.5em}
        \item{As $T \to 0K$}, all states at energies below $E_F$  will be filled, and all states at energies above $E_F$ will be empty.
        \item{For $T > 0\;K$ and $E \geq E_F + 3kT$, the Fermi function, or filled-state probability decays exponentially to zero with increasing energy; most states at energies $3kT$ or more above $E_F$ will be empty.}
        \item{For $T > 0\;K$ and $E \leq E_F - 3kT$, the probability that a given state will be \emph{empty} decays exponentially to zero with decreasing energy; most states at energies $3kT$ or more below $E_F$ will be filled.}
        \item{The Fermi Function applies \emph{only} under equilibrium conditions.}
        \item{The Fermi function is universal in the sense that it applies to all materials; insulators, metals, and semiconductors.  It is a statistical functions associated with electrons in general.}
    \end{itemize}
%%%%%%%%%%%%%%%%%
\vspace{0.5cm}
%%%%%%%%%%%%%%%%%
    \textbf{Fermi Level} - The thermodynamic work required to add one electron to a solid-state body.  When the Fermi level is positioned in the upper half of the band gap (or higher), the electron distribution greatly outweighs the hole distribution.
    %%%%%%%%%%%%%%%%%
    \vspace{0.15cm}
    %%%%%%%%%%%%%%%%%
    \begin{itemize}
        \setlength\itemsep{0.5em}
        \item{Under equilibrium conditions, $\frac{dE_F}{dx} = \frac{dE_F}{dy} = \frac{dE_F}{dz} = 0$; i.e. the Fermi level inside a material or a group of materials in intimate contact is invariant as a function of position.}
    \end{itemize}
%%%%%%%%%%%%%%%%%
\vspace{0.5cm}
%%%%%%%%%%%%%%%%%
    \textbf{Fermi Level Pinning} - In most real $MS-diodes$ (both $P$\emph{-type} and $N$\emph{-type}), $\Phi_B \not = \Phi_M - \chi$.  In the majority of semiconductors, surface charges tend to fix or "pin" the equilibrium Fermi level at a specific energy within the surface band gap.  Because of this pinning effect, the observed barrier height normally varies only slightly with the metal used to fabricate the diode.
    %%%%%%%%%%%%%%%%%
    \vspace{0.15cm}
    %%%%%%%%%%%%%%%%%
    \begin{itemize}
        \setlength\itemsep{0.5em}
        \item{There are high densities of energy states in the band gap at the metal-semiconductor interface.}
        \item{Some of these states are acceptor like, and may be neutral or negative.}
        \item{Some of these states are donor like, and may be neutral or positive.}
        \item{The net charge is zero when the Fermi level at the interface is around the middle of the silicon band gap.}
        \item{The result is that for any $\psi_M \not \approx 4.6\;V$, there is a dipole at the interface that prevents the barrier height from moving very far from around $0.7\;V$.}
    \end{itemize}
%%%%%%%%%%%%%%%%%
\vspace{0.5cm}
%%%%%%%%%%%%%%%%%
    \textbf{Flat-Band Condition} - A condition where the energy band ($E_C$ and $E_V$) of the substrate is flat at the $Si-{SiO}_2$ interface.
    %%%%%%%%%%%%%%%%%
    \vspace{0.15cm}
    %%%%%%%%%%%%%%%%%
    \begin{itemize}
        \setlength\itemsep{0.5em}
        \item The flat-band voltage is the difference between the Fermi levels at the two terminals:
        \begin{equation}
            V_{FB} = \psi_g - \psi_s
        \end{equation}
        Where $\psi_g$ and $\psi_s$ are the gate work function and the semiconductor work function, respectively, in volts.
    \end{itemize}
%%%%%%%%%%%%%%%%%
\vspace{0.5cm}
%%%%%%%%%%%%%%%%%
    \textbf{Freeze Out} - A failure to ionize due to too low of a temperature.
    \begin{itemize}
        \setlength\itemsep{0.5em}
        \item $kT < E_{ionization}$
    \end{itemize}
%%%%%%%%%%%%%%%%%
\vspace{0.5cm}
%%%%%%%%%%%%%%%%%
    \textbf{Frequency Response} - A system's dependence on signal frequency of the output–input ratio of an amplifier or other device.  The frequency response of a system is the quantitative measure of the magnitude and phase of the output as a function of input frequency.
%%%%%%%%%%%%%%%%%%%%%%%%%%%%%%%%%%%%%%%%%%%%%%%%%%%%%%%%%%%%%%%%%%%%%%%%%%%%%%%%%%%%%%
%                                   SECTION G                                        %
%%%%%%%%%%%%%%%%%%%%%%%%%%%%%%%%%%%%%%%%%%%%%%%%%%%%%%%%%%%%%%%%%%%%%%%%%%%%%%%%%%%%%%
\section{G}
    \textbf{Gain-Bandwidth Product} - The product of an amplifier's bandwidth and the gain at which the bandwidth is measured.
    %%%%%%%%%%%%%%%%%
    \vspace{0.15cm}
    %%%%%%%%%%%%%%%%%
    \textbf{Gate Induced Drain Leakage} - A parasitic current that escapes through the portion of the thin gate oxide in MOSFETs that partially overlaps with the drain. GIDL occurs at drain voltages that are extremely small and much less than junction breakdown voltages.
    %%%%%%%%%%%%%%%%%
    \vspace{0.15cm}
    %%%%%%%%%%%%%%%%%
    \begin{itemize}
        \setlength\itemsep{0.5em}
        \item{Occurs when $V_{dg}$ is large enough to create band-bending that facilitates band-to-band tunneling between the drain and the gate, because the channel/drain junction is heavily reverse-biased, which causes the barrier to be narrowed.}
        \item{GIDL current increases with increasing $V_d$ and decreasing $V_g$.}
    \end{itemize}
%%%%%%%%%%%%%%%%%
\vspace{0.5cm}
%%%%%%%%%%%%%%%%%
    \textbf{Generation} - A process whereby electrons and holes are created.
%%%%%%%%%%%%%%%%%%%%%%%%%%%%%%%%%%%%%%%%%%%%%%%%%%%%%%%%%%%%%%%%%%%%%%%%%%%%%%%%%%%%%%
%                                   SECTION H                                        %
%%%%%%%%%%%%%%%%%%%%%%%%%%%%%%%%%%%%%%%%%%%%%%%%%%%%%%%%%%%%%%%%%%%%%%%%%%%%%%%%%%%%%%
\section{H}
    \textbf{Hole} - The empty state in the valence band left behind by an electron that has moved to the conduction band.\\
    %%%%%%%%%%%%%%%%%
    \vspace{0.15cm}
    %%%%%%%%%%%%%%%%%
    \textbf{Hysteresis} - The phenomenon in which the value of a physical property lags behind changes in the effect causing it, as for instance when magnetic induction lags behind the magnetizing force.
%%%%%%%%%%%%%%%%%
\vspace{0.5cm}
%%%%%%%%%%%%%%%%%
\noindent
    \textbf{Hot-Point Probe Measurement} - A common technique for rapidly determining whether a semiconductor is $P$\emph{-type} or $N$\emph{-type}.
%%%%%%%%%%%%%%%%%%%%%%%%%%%%%%%%%%%%%%%%%%%%%%%%%%%%%%%%%%%%%%%%%%%%%%%%%%%%%%%%%%%%%%
%                                   SECTION I                                        %
%%%%%%%%%%%%%%%%%%%%%%%%%%%%%%%%%%%%%%%%%%%%%%%%%%%%%%%%%%%%%%%%%%%%%%%%%%%%%%%%%%%%%%
\section{I}
    \textbf{Impact Ionization} - At sufficiently high reverse bias the minority electrons may gain enough kinetic energy such that as they collide with the lattice it creates electron hole pairs.

%%%%%%%%%%%%%%%%%
\vspace{0.5cm}
%%%%%%%%%%%%%%%%%
\noindent
    \textbf{Impurity Scattering} - The scattering of charge carriers by ionization in the lattice.
    %%%%%%%%%%%%%%%%%
    \vspace{0.15cm}
    %%%%%%%%%%%%%%%%%
    \begin{itemize}
        \setlength\itemsep{0.5em}
        \item{Does not go up at higher temperatures.}
    \end{itemize}
%%%%%%%%%%%%%%%%%
\vspace{0.5cm}
%%%%%%%%%%%%%%%%%
    \textbf{Indirect Bandgap} - Differs from a direct bandgap in that now the carriers must be assisted by a phonon in order to pass through an intermediate state before emmitting a photon, or recombining.
    \begin{itemize}
        \setlength\itemsep{0.5em}
        \item{Dominant R-G mechanism is R-G center.}
    \end{itemize}
%%%%%%%%%%%%%%%%%
\vspace{0.5cm}
%%%%%%%%%%%%%%%%%
    \textbf{Ingot (semiconductor)} - An oblong bar of silicon, which is used to cut and polish into chip wafers.
    
%%%%%%%%%%%%%%%%%
\vspace{0.5cm}
%%%%%%%%%%%%%%%%%
    \textbf{Intrinsic Fermi Level} - A dashed line on a band diagram representing the expected positioning of the Fermi level if the material was intrinsic, and it serves as a reference energy level dividing the upper and lower halves of the band gap.
    
%%%%%%%%%%%%%%%%%
\vspace{0.5cm}
%%%%%%%%%%%%%%%%%
\noindent
    \textbf{Intrinsic Semiconductor} - An extremely pure semiconductor sample containing an insignificant amount of impurity atoms.  Its properties are native to the material and not caused by external additives.
    %%%%%%%%%%%%%%%%%
    \vspace{0.15cm}
    %%%%%%%%%%%%%%%%%
    \begin{itemize}
        \setlength\itemsep{0.5em}
        \item{The electrons and holes in an intrinsic semiconductor are equal because carriers within a very pure material can only be created in pairs.}
    \end{itemize}
%%%%%%%%%%%%%%%%%%%%%%%%%%%%%%%%%%%%%%%%%%%%%%%%%%%%%%%%%%%%%%%%%%%%%%%%%%%%%%%%%%%%%%
%                                   SECTION J                                        %
%%%%%%%%%%%%%%%%%%%%%%%%%%%%%%%%%%%%%%%%%%%%%%%%%%%%%%%%%%%%%%%%%%%%%%%%%%%%%%%%%%%%%%
\section{J}
%%%%%%%%%%%%%%%%%%%%%%%%%%%%%%%%%%%%%%%%%%%%%%%%%%%%%%%%%%%%%%%%%%%%%%%%%%%%%%%%%%%%%%
%                                   SECTION K                                        %
%%%%%%%%%%%%%%%%%%%%%%%%%%%%%%%%%%%%%%%%%%%%%%%%%%%%%%%%%%%%%%%%%%%%%%%%%%%%%%%%%%%%%%
\section{K}
%%%%%%%%%%%%%%%%%%%%%%%%%%%%%%%%%%%%%%%%%%%%%%%%%%%%%%%%%%%%%%%%%%%%%%%%%%%%%%%%%%%%%%
%                                   SECTION L                                        %
%%%%%%%%%%%%%%%%%%%%%%%%%%%%%%%%%%%%%%%%%%%%%%%%%%%%%%%%%%%%%%%%%%%%%%%%%%%%%%%%%%%%%%
\section{L}
    \textbf{Lattice Scattering} - The scattering of ions by interaction with atoms in a lattice.[1] This effect can be qualitatively understood as phonons colliding with charge carriers.
    %%%%%%%%%%%%%%%%%
    \vspace{0.15cm}
    %%%%%%%%%%%%%%%%%
    \begin{itemize}
        \setlength\itemsep{0.5em}
        \item{When the wavelength of the electrons is larger than the crystal spacing, the electrons will propagate freely throughout the metal without collision.}
    \end{itemize}
%%%%%%%%%%%%%%%%%
\vspace{0.5cm}
%%%%%%%%%%%%%%%%%
    \textbf{Law of the Junction} - The number of electrons (or holes) crossing over from the $n-side$ to the $p-side$ increases exponentially as a function of the forward-bias voltage.
    %%%%%%%%%%%%%%%%%
    \vspace{0.15cm}
    %%%%%%%%%%%%%%%%%
    \begin{itemize}
        \setlength\itemsep{0.5em}
        \item{
            \begin{equation}
                \large{p_{p-side_{depeletion\;edge}} = p_{n-side_{depeletion\;edge}} \cdot e^{(\phi_{bi} - V_{applied}) / \phi_T}}
            \end{equation}
            }
        \item{The sum of the in-flowing current is equal to the sum of out-flowing current.}
        \item{$E_{Fp}$ and $E_{Fn}$ do not change across the depletion region.}
    \end{itemize}
%%%%%%%%%%%%%%%%%%%%%%%%%%%%%%%%%%%%%%%%%%%%%%%%%%%%%%%%%%%%%%%%%%%%%%%%%%%%%%%%%%%%%%
%                                   SECTION M                                        %
%%%%%%%%%%%%%%%%%%%%%%%%%%%%%%%%%%%%%%%%%%%%%%%%%%%%%%%%%%%%%%%%%%%%%%%%%%%%%%%%%%%%%%
\section{M}
    \textbf{Majority Carrier} - The most abundant carrier in a given semiconductor sample; electrons in an $N$\emph{-type} material, holes in a $P$\emph{-type} material.

%%%%%%%%%%%%%%%%%
\vspace{0.5cm}
%%%%%%%%%%%%%%%%%
\noindent
    \textbf{MIM Capacitor} - A capacitor that consists of parallel plates formed by two metal planes separated by a thin dielectric.

%%%%%%%%%%%%%%%%%
\vspace{0.5cm}
%%%%%%%%%%%%%%%%%
\noindent
    \textbf{Minority Carrier} - The least abundant carrier in a given semiconductor sample; holes in an $N$\emph{-type} material, electrons in a $P$\emph{-type} material.

%%%%%%%%%%%%%%%%%
\vspace{0.5cm}
%%%%%%%%%%%%%%%%%
\noindent
    \textbf{Minority Carrier Injection} -

%%%%%%%%%%%%%%%%%
\vspace{0.5cm}
%%%%%%%%%%%%%%%%%
\noindent
    \textbf{Mobility} - A measurement of the ease of carrier motion in a crystal; varies inversely with the amount of scattering taking place within a semiconductor; dependent on both doping and temperature.
    %%%%%%%%%%%%%%%%%
    \vspace{0.15cm}
    %%%%%%%%%%%%%%%%%
    \begin{itemize}
        \setlength\itemsep{0.5em}
        \item{Increasing the motion-impeding collisions within a crystal decreases the mobility of the carriers (i.e. the mean free time between collisions).}
        \item{The mobility also varies inversely with the effective carrier mass; lighter carriers move more readily.}
        \item{Lattice scattering decreases with decreasing $T$, whereas ionized impurity scattering increases with decreasing $T$.}
        \item{Ionized impurity scattering becomes a larger and larger percentage of the overall scattering as the temperature is decreased.}
    \end{itemize}
%%%%%%%%%%%%%%%%%
\vspace{0.5cm}
%%%%%%%%%%%%%%%%%
    \textbf{MOS Capacitor} - A simple two-terminal device composed of a thin ($0.01\;\mu m - 1.0\;\mu m$) ${SiO}_2$ layer sandwiched between a silicon substrate and a metallic field plate.  The ideal MOS structure has the following properties:
    %%%%%%%%%%%%%%%%%
    \vspace{0.15cm}
    %%%%%%%%%%%%%%%%%
    \begin{itemize}
        \setlength\itemsep{0.5em}
        \item{The metallic gate is sufficiently thick so that it can be considered an equipotential region under both AC and DC biasing conditions.}
        \item{The oxide is a \emph{perfect insulator} with \emph{zero current} flowing through the oxide layer under \emph{all} static biasing conditions.}
        \item{There are no charge centers located in the oxide, or at the oxide-semiconductor interface.}
        \item{The semiconductor is uniformly doped.}
        \item{The semiconductor is sufficiently thick so that, regardless of the applied gate potential, a field-free region (the body, bulk, or substrate) is encountered before reaching the back contact.}
        \item{An \emph{ohmic} contact has been established between the semiconductor and the metal on the back side of the device.}
        \item{The $MOS-C$ is a one-dimensional device taken to be a function of only the $x$-coordinate.}
    \end{itemize}
%%%%%%%%%%%%%%%%%
\vspace{0.5cm}
%%%%%%%%%%%%%%%%%
    \textbf{MOS Transistor} - 
    %%%%%%%%%%%%%%%%%
    \vspace{0.15cm}
    %%%%%%%%%%%%%%%%%
    \begin{itemize}
        \setlength\itemsep{0.5em}
        \item{}
    \end{itemize}
%%%%%%%%%%%%%%%%%
\vspace{0.5cm}
%%%%%%%%%%%%%%%%%
    \textbf{M-S Junction} - A type of electrical junction in which a metal comes in close contact with a semiconductor material.
    %%%%%%%%%%%%%%%%%
    \vspace{0.15cm}
    %%%%%%%%%%%%%%%%%
    \begin{itemize}
        \setlength\itemsep{0.5em}
        \item{M–S junctions can either be rectifying or non-rectifying.}
        \item{The rectifying metal–semiconductor junction forms a Schottky barrier, making a device known as a Schottky diode.}
        \item{The non-rectifying junction is called an ohmic contact.}
    \end{itemize}
%%%%%%%%%%%%%%%%%%%%%%%%%%%%%%%%%%%%%%%%%%%%%%%%%%%%%%%%%%%%%%%%%%%%%%%%%%%%%%%%%%%%%%
%                                   SECTION N                                        %
%%%%%%%%%%%%%%%%%%%%%%%%%%%%%%%%%%%%%%%%%%%%%%%%%%%%%%%%%%%%%%%%%%%%%%%%%%%%%%%%%%%%%%
\section{N}
    \textbf{$N$\emph{-type} Material} - A donor-doped material containing more electrons than holes.
%%%%%%%%%%%%%%%%%%%%%%%%%%%%%%%%%%%%%%%%%%%%%%%%%%%%%%%%%%%%%%%%%%%%%%%%%%%%%%%%%%%%%%
%                                   SECTION O                                        %
%%%%%%%%%%%%%%%%%%%%%%%%%%%%%%%%%%%%%%%%%%%%%%%%%%%%%%%%%%%%%%%%%%%%%%%%%%%%%%%%%%%%%%
\section{O}
    \textbf{Ohmic Contact} - A non-rectifying electrical junction: a junction between two conductors that has a linear current–voltage (I-V) curve as with Ohm's law.
    %%%%%%%%%%%%%%%%%
    \vspace{0.15cm}
    %%%%%%%%%%%%%%%%%
    \begin{itemize}
        \setlength\itemsep{0.5em}
        \item{Low resistance ohmic contacts are used to allow charge to flow easily in both directions between the two conductors, without blocking due to rectification or excess power dissipation due to voltage thresholds.}
        \item{\emph{Boundary Condition of an Ohmic Contact} : The voltage across an ideal ohmic contact is zero.  This means that the Fermi level cannot deviate from its equilibrium position, and therefore $n'=p'=0$ at an ideal ohmic contact.}
    \end{itemize}
%%%%%%%%%%%%%%%%%
\vspace{0.5cm}
%%%%%%%%%%%%%%%%%
    \textbf{Operating Point} - Also known as bias point, quiescent point, or Q-point, it is the DC voltage or current at a specified terminal of an active device (a transistor or vacuum tube) with no input signal applied.
    %%%%%%%%%%%%%%%%%
    \vspace{0.15cm}
    %%%%%%%%%%%%%%%%%
    \begin{itemize}
        \setlength\itemsep{0.5em}
        \item{null}
    \end{itemize}
%%%%%%%%%%%%%%%%%
\vspace{0.5cm}
%%%%%%%%%%%%%%%%%
    \textbf{Optical Phonon} - A high energy (frequency) phonon.  They are out-of-phase movements of the atoms in the lattice, one atom moving to the left, and its neighbor to the right.
    %%%%%%%%%%%%%%%%%
    \vspace{0.15cm}
    %%%%%%%%%%%%%%%%%
    \begin{itemize}
        \setlength\itemsep{0.5em}
        \item{Occur at higher temperatures and with higher energy.}
    \end{itemize}
%%%%%%%%%%%%%%%%%
\vspace{0.5cm}
%%%%%%%%%%%%%%%%%
    \textbf{One-Sided Junction} - A PN junction where one region is highly doped in comparison to the doping of the other region. In this junction, the concentration of one side impurity is considered.
    %%%%%%%%%%%%%%%%%
    \vspace{0.15cm}
    %%%%%%%%%%%%%%%%%
    \begin{itemize}
        \setlength\itemsep{0.5em}
        \item{null}
    \end{itemize}
%%%%%%%%%%%%%%%%%
\vspace{0.5cm}
%%%%%%%%%%%%%%%%%
    \textbf{Overdrive Voltage} - The voltage between transistor gate and source in excess of the threshold voltage.
    %%%%%%%%%%%%%%%%%
    \vspace{0.15cm}
    %%%%%%%%%%%%%%%%%
    \begin{itemize}
        \setlength\itemsep{0.5em}
        \item{Typically referred to in the context of MOSFET transistors.}
        \item{Also known as "excess gate voltage" or "effective voltage."}
        \item{Can be found using the following equation:}
        \begin{equation}
            V_{OV} = V_{GS} - V_{TH}
        \end{equation}
    \end{itemize}
%%%%%%%%%%%%%%%%%%%%%%%%%%%%%%%%%%%%%%%%%%%%%%%%%%%%%%%%%%%%%%%%%%%%%%%%%%%%%%%%%%%%%%
%                                   SECTION P                                        %
%%%%%%%%%%%%%%%%%%%%%%%%%%%%%%%%%%%%%%%%%%%%%%%%%%%%%%%%%%%%%%%%%%%%%%%%%%%%%%%%%%%%%%
\section{P}
    \textbf{$P$\emph{-type} Material} - An acceptor-doped material containing more holes than electrons.

%%%%%%%%%%%%%%%%%
\vspace{0.5cm}
%%%%%%%%%%%%%%%%%
\noindent
    \textbf{Parasitic capacitance} - An unavoidable and usually unwanted capacitance that exists between the parts of an electronic component or circuit simply because of their proximity to each other.

%%%%%%%%%%%%%%%%%
\vspace{0.5cm}
%%%%%%%%%%%%%%%%%
\noindent
    \textbf{Pauli Exclusion Principle} - States that electrons are restricted to single occupancy in allowed energy states.

%%%%%%%%%%%%%%%%%
\vspace{0.5cm}
%%%%%%%%%%%%%%%%%
\noindent
    \textbf{Phonon} - Vibrations of the lattice.
    
%%%%%%%%%%%%%%%%%
\vspace{0.5cm}
%%%%%%%%%%%%%%%%%
\noindent
    \textbf{Photolithography} - A general term used for techniques that use light to produce minutely patterned thin films of suitable materials over a substrate, such as a silicon wafer, to protect selected areas of it during subsequent etching, deposition, or implantation operations.
    
%%%%%%%%%%%%%%%%%
\vspace{0.5cm}
%%%%%%%%%%%%%%%%%
\noindent
    \textbf{Photoresist} - A light-sensitive material used in several processes, such as photolithography and photoengraving, to form a patterned coating on a surface.
%%%%%%%%%%%%%%%%%%%%%%%%%%%%%%%%%%%%%%%%%%%%%%%%%%%%%%%%%%%%%%%%%%%%%%%%%%%%%%%%%%%%%%
%                                   SECTION Q                                        %
%%%%%%%%%%%%%%%%%%%%%%%%%%%%%%%%%%%%%%%%%%%%%%%%%%%%%%%%%%%%%%%%%%%%%%%%%%%%%%%%%%%%%%
\section{Q}
    \textbf{Quantum Mechanical Tunneling} - A phenomenon where a wavefunction can propagate through a potential barrier.
%%%%%%%%%%%%%%%%%
\vspace{0.15cm}
%%%%%%%%%%%%%%%%%
    \begin{itemize}
        \setlength\itemsep{0.5em}
        \item{The transmission through the barrier can be finite and depends exponentially on the barrier height and barrier width.}
        \item{The wavefunction may disappear on one side and reappear on the other side. The wavefunction and its first derivative are continuous.}
        \item{In steady-state, the probability flux in the forward direction is spatially uniform. No particle or wave is lost.}
        \item{Tunneling occurs with barriers of thickness around $1-3\;nm$ and smaller.}
    \end{itemize}
%%%%%%%%%%%%%%%%%
\vspace{0.5cm}
%%%%%%%%%%%%%%%%%
    \textbf{Quality (Q) Factor} - A dimensionless parameter that describes how underdamped an oscillator or resonator is. It is defined as the ratio of the initial energy stored in the resonator to the energy lost in one radian of the cycle of oscillation.
    \begin{itemize}
        \setlength\itemsep{0.5em}
        \item{Higher Q indicates a lower rate of energy loss and the oscillations die out more slowly.}
        \item{A pendulum suspended from a high-quality bearing, oscillating in air, has a high Q, while a pendulum immersed in oil has a low one.}
        \item{Resonators with high quality factors have low damping, so that they ring or vibrate longer.}
    \end{itemize}
%%%%%%%%%%%%%%%%%%%%%%%%%%%%%%%%%%%%%%%%%%%%%%%%%%%%%%%%%%%%%%%%%%%%%%%%%%%%%%%%%%%%%%
%                                   SECTION R                                        %
%%%%%%%%%%%%%%%%%%%%%%%%%%%%%%%%%%%%%%%%%%%%%%%%%%%%%%%%%%%%%%%%%%%%%%%%%%%%%%%%%%%%%%
\section{R}
    \textbf{Recombination} - A process whereby electrons and holes (carriers) are annihilated or destroyed.

%%%%%%%%%%%%%%%%%
\vspace{0.5cm}
%%%%%%%%%%%%%%%%%
\noindent
    \textbf{Rectifier} - An electrical device that converts alternating current (AC), which periodically reverses direction, to direct current (DC), which flows in only one direction. The reverse operation is performed by the inverter.
    %%%%%%%%%%%%%%%%%
    \vspace{0.15cm}
    %%%%%%%%%%%%%%%%%
    \begin{itemize}
        \setlength\itemsep{0.5em}
        \item{The process is known as rectification, since it "straightens" the direction of current.}
    \end{itemize}
%%%%%%%%%%%%%%%%%
\vspace{0.5cm}
%%%%%%%%%%%%%%%%%
    \textbf{Resistivity} - A measure of a material's inherent resistance to current flow.  Quantitatively, the proportionality constant between the electric field impressed across a homogeneous material and the total particle current per unit area flowing in the material.

%%%%%%%%%%%%%%%%%
\vspace{0.5cm}
%%%%%%%%%%%%%%%%%
\noindent
    \textbf{R-G Centers} - Special locations within the semiconductor, which are lattice defects or special impurity atoms such as gold in $Si$.

%%%%%%%%%%%%%%%%%
\vspace{0.5cm}
%%%%%%%%%%%%%%%%%
\noindent
    \textbf{R-G Processes} - The means whereby the carrier excess or deficit inside the semiconductor is stabilized (if the perturbation is maintained) or eliminated (if the perturbation is removed.
    %%%%%%%%%%%%%%%%%
    \vspace{0.15cm}
    %%%%%%%%%%%%%%%%%
    \begin{itemize}
        \setlength\itemsep{0.5em}
        \item{\textbf{Band-to-Band Recombination} - The direct annihilation of a conduction band electron and a valence band hole.  The excess energy released is typically a photon.}
        \item{\textbf{R-G Center Recombination} - Also called \emph{indirect combination}.  Due to the R-G centers, allowed electronic levels near the center of the band gap are introduced.  First, one type of carrier strays into the vicinity of an R-G center and is caught by the potential well associated with the center, loses energy, and is trapped.  Then, the opposite type of carrier comes along and is attracted to the already trapped carrier, loses energy, and annihilates.  Typically produces thermal energy or lattice vibrations.}
        \item{\textbf{Auger Recombination} - }
        \item{\textbf{Impact Ionization} - The inverse of Auger recombination.}
    \end{itemize}
%%%%%%%%%%%%%%%%%%%%%%%%%%%%%%%%%%%%%%%%%%%%%%%%%%%%%%%%%%%%%%%%%%%%%%%%%%%%%%%%%%%%%%
%                                   SECTION S                                        %
%%%%%%%%%%%%%%%%%%%%%%%%%%%%%%%%%%%%%%%%%%%%%%%%%%%%%%%%%%%%%%%%%%%%%%%%%%%%%%%%%%%%%%
\section{S}
    \textbf{Scattering} - 
    %%%%%%%%%%%%%%%%%
    \vspace{0.15cm}
    %%%%%%%%%%%%%%%%%
    \begin{itemize}
        \setlength\itemsep{0.5em}
        \item{Lattice scattering}
        \item{Ionized impurity scattering}
    \end{itemize}
%%%%%%%%%%%%%%%%%
\vspace{0.5cm}
%%%%%%%%%%%%%%%%%
    \textbf{Schottky Barrier} - A potential energy barrier for electrons formed at a metal–semiconductor junction.
    %%%%%%%%%%%%%%%%%
    \vspace{0.15cm}
    %%%%%%%%%%%%%%%%%
    \begin{itemize}
        \setlength\itemsep{0.5em}
        \item{Schottky barriers have rectifying characteristics, suitable for use as a diode.}
        \item{The Schottky barrier height, denoted by $\phi_B$, is a function of the metal and semiconductor.}
        \begin{itemize}
            \setlength\itemsep{0.5em}
            \item{There are actually two energy barriers.}
            \item{$q\phi_{B_n}$ is the barrier against electron flow between the metal and the $N$\emph{-type} semiconductor.}
            \item{$q\phi_{B_p}$ is the barrier against hole flow between the metal and the $P$\emph{-type} semiconductor.}
            \item{The sum of both barrier heights for any type of material is equal to its bandgap energy.  For silicon:}
            \begin{equation}
                \phi_{B_n} + \phi_{B_p} \approx E_g
            \end{equation}
            \item{There is a trend that $\phi_{B_n}$ and $\phi_{B_p}$ increase with an increasing metal work function.  This can be explained by:}
                \begin{equation}
                    \phi_{B_n} = \psi_M - \chi_{Si}
                \end{equation}
                Where $\psi_M$ is the metal work function, and $\chi_{Si}$ is the electron affinity.
        \end{itemize}
    \end{itemize}
%%%%%%%%%%%%%%%%%
\vspace{0.5cm}
%%%%%%%%%%%%%%%%%
    \textbf{Synchronous Rectifier} - A circuit that emulates a diode, allowing current to pass in one direction but not the other without the losses associated with junction or Schottky devices. The circuit comprises a pass-element (most often a power MOSFET), a sense element, a sense-signal conditioner, and a driver.
    \begin{itemize}
        \setlength\itemsep{0.5em}
        \item{In the MOSFET application, the rectifier would have a large channel width in order to conduct large currents.}
    \end{itemize}
%%%%%%%%%%%%%%%%%%%%%%%%%%%%%%%%%%%%%%%%%%%%%%%%%%%%%%%%%%%%%%%%%%%%%%%%%%%%%%%%%%%%%%
%                                   SECTION T                                        %
%%%%%%%%%%%%%%%%%%%%%%%%%%%%%%%%%%%%%%%%%%%%%%%%%%%%%%%%%%%%%%%%%%%%%%%%%%%%%%%%%%%%%%
\section{T}
    \textbf{Thermal Equilibrium} - See equilibrium.
%%%%%%%%%%%%%%%%%
\vspace{0.15cm}
%%%%%%%%%%%%%%%%%

    \textbf{Thermal Motion} - The random motions of molecules, atoms, electrons or other subatomic particles.
%%%%%%%%%%%%%%%%%
\vspace{0.15cm}
%%%%%%%%%%%%%%%%%
    \begin{itemize}
        \setlength\itemsep{0.5em}
        \item{Thermal motion in a doped semiconductor averages out to zero on a macroscopic scale, and does not contribute to current transport.}
    \end{itemize}
%%%%%%%%%%%%%%%%%
\vspace{0.5cm}
%%%%%%%%%%%%%%%%%
    \textbf{Thermal Runaway} - A process that is accelerated by increased temperature, in turn releasing energy that further increases temperature.
    %%%%%%%%%%%%%%%%%
    \vspace{0.15cm}
    %%%%%%%%%%%%%%%%%
    \begin{itemize}
        \setlength\itemsep{0.5em}
        \item{Thermal runaway occurs in situations where an increase in temperature changes the conditions in a way that causes a further increase in temperature, often leading to a destructive result.}
        \item{It is a kind of uncontrolled positive feedback.}
    \end{itemize}
%%%%%%%%%%%%%%%%%
\vspace{0.15cm}
%%%%%%%%%%%%%%%%%
    \textbf{Thermal Voltage} - The voltage produced within a $PN$-junction due to temperature.

%%%%%%%%%%%%%%%%%
\vspace{0.5cm}
%%%%%%%%%%%%%%%%%
    \textbf{Thermionic Emission Current} - The current resulting from majority carrier electron or hole injection over the potential barrier in an MS diode.
    %%%%%%%%%%%%%%%%%
    \vspace{0.15cm}
    %%%%%%%%%%%%%%%%%
    \begin{itemize}
        \setlength\itemsep{0.5em}
        \item{Thermal runaway occurs in situations where an increase in temperature changes the conditions in a way that causes a further increase in temperature, often leading to a destructive result.}
        \item{It is a kind of uncontrolled positive feedback.}
    \end{itemize}
%%%%%%%%%%%%%%%%%
\vspace{0.5cm}
%%%%%%%%%%%%%%%%%
    \textbf{Transconductance} - The trans(fer) conductance is the electrical characteristic relating the current through the output of a device to the voltage across the input of a device.
    %%%%%%%%%%%%%%%%%
    \vspace{0.15cm}
    %%%%%%%%%%%%%%%%%
    \begin{itemize}
        \setlength\itemsep{0.5em}
        \item{It is often denoted as a conductance, with a subscript, m, for mutual, defined as follows:}
            \begin{equation}
                g_m = \frac{\Delta I_{out}}{\Delta V_{in}}
            \end{equation}
        \item{For small signal models:}
            \begin{equation}
                g_m = \frac{\partial i_{out}}{\partial v_{in}}
            \end{equation}
        \item{The transconductance for the bipolar transistor can be expressed as:}
            \begin{equation}
                g_m = \frac{I_C}{V_T}
            \end{equation}
            $I_C$ is the DC collector current at the quiescent point, and $V_T$ is the thermal voltage ($\approx 0.259\;V$).
        \item{The transconductance for the MOSFET can be expressed as:}
            \begin{equation}
                g_m = \frac{2I_D}{V_{OV}}
            \end{equation}
            $I_D$ is the DC drain current at the bias point, and $V_{OV}$ is the overdrive voltage.
        \item{For vacuum tubes, transconductance is defined as the change in the plate (anode) current divided by the corresponding change in the grid/cathode voltage, with a constant plate(anode) to cathode voltage.  Defined as follows:}
            \begin{equation}
                g_m = \frac{\mu}{r_p}
            \end{equation}
            $\mu$ is the gain, and $r_p$ is the plate resistance.
        \item{A transconductance amplifier puts out a current proportional to its input voltage.}
    \end{itemize}
%%%%%%%%%%%%%%%%%
\vspace{0.5cm}
%%%%%%%%%%%%%%%%%
    \textbf{Transresistance} - The trans(fer) resistance refers to the ratio between a change of the voltage at two output points and a related change of current through two input points.
    \begin{itemize}
        \setlength\itemsep{0.5em}
        \item{Also denoted with a subscript, m, for mutual, defined as follows:}
            \begin{equation}
                r_m = \frac{\Delta V_{out}}{\Delta I_{in}}
            \end{equation}
        \item{For small signal models:}
            \begin{equation}
                r_m = \frac{\delta v_{out}}{\delta i_{in}}
            \end{equation}
        \item{A transresistance amplifier outputs a voltage proportional to its input current. The transresistance amplifier is often referred to as a transimpedance amplifier, especially by semiconductor manufacturers.}
        \item{The term for a transresistance amplifier in network analysis is current controlled voltage source (CCVS).}
    \end{itemize}
%%%%%%%%%%%%%%%%%%%%%%%%%%%%%%%%%%%%%%%%%%%%%%%%%%%%%%%%%%%%%%%%%%%%%%%%%%%%%%%%%%%%%%
%                                   SECTION U                                        %
%%%%%%%%%%%%%%%%%%%%%%%%%%%%%%%%%%%%%%%%%%%%%%%%%%%%%%%%%%%%%%%%%%%%%%%%%%%%%%%%%%%%%%
\section{U}
%%%%%%%%%%%%%%%%%%%%%%%%%%%%%%%%%%%%%%%%%%%%%%%%%%%%%%%%%%%%%%%%%%%%%%%%%%%%%%%%%%%%%%
%                                   SECTION V                                        %
%%%%%%%%%%%%%%%%%%%%%%%%%%%%%%%%%%%%%%%%%%%%%%%%%%%%%%%%%%%%%%%%%%%%%%%%%%%%%%%%%%%%%%
\section{V}
    \textbf{Vacuum Level} - The minimum energy (typically denoted as $E_0$) and electron must possess to completely free itself from a material.

%%%%%%%%%%%%%%%%%
\vspace{0.5cm}
%%%%%%%%%%%%%%%%%
    \textbf{Varactor} - Also known as a varicap diode, a varactor is a type of diode designed to exploit the voltage-dependent capacitance of a reverse-biased $PN$-junction.  Varactors are used as voltage-controlled capacitors. They are commonly used in voltage-controlled oscillators, parametric amplifiers, and frequency multipliers.

%%%%%%%%%%%%%%%%%
\vspace{0.5cm}
%%%%%%%%%%%%%%%%%
\noindent
    \textbf{Valence Band} - The lower band of allowed energy states in $Si$.

%%%%%%%%%%%%%%%%%
\vspace{0.5cm}
%%%%%%%%%%%%%%%%%
\noindent
    \textbf{Velocity Saturation} - The maximum velocity a charge carrier in a semiconductor, generally an electron, attains in the presence of very high electric fields.
    %%%%%%%%%%%%%%%%%
    \vspace{0.15cm}
    %%%%%%%%%%%%%%%%%
    \begin{itemize}
        \setlength\itemsep{0.5em}
        \item{As the applied electric field increases from that point, the carrier velocity no longer increases, because the carriers lose energy through increased levels of interaction with the lattice by emitting phonons, and even photons, as soon as the carrier energy is large enough to do so.}
    \end{itemize}
%%%%%%%%%%%%%%%%%%%%%%%%%%%%%%%%%%%%%%%%%%%%%%%%%%%%%%%%%%%%%%%%%%%%%%%%%%%%%%%%%%%%%%
%                                   SECTION W                                        %
%%%%%%%%%%%%%%%%%%%%%%%%%%%%%%%%%%%%%%%%%%%%%%%%%%%%%%%%%%%%%%%%%%%%%%%%%%%%%%%%%%%%%%
\section{W}
    \textbf{Work Function} - The distance from vacuum level to the Fermi level of a material.
    %%%%%%%%%%%%%%%%%
    \vspace{0.15cm}
    %%%%%%%%%%%%%%%%%
    \begin{itemize}
        \setlength\itemsep{0.5em}
        \item{$\Phi_s \triangleq (E_0 - {E_C}_{semiconductor}) + (E_C - E_F) = \chi + (E_C - E_F)$ is the work function of a semiconductor.  The affinity is used because the conduction band edge is not a constant in semiconductors.}
        \item{$\Phi_m \triangleq E_0 - {E_F}_{metal}$ is the work function of a metal.}
    \end{itemize}
%%%%%%%%%%%%%%%%%%%%%%%%%%%%%%%%%%%%%%%%%%%%%%%%%%%%%%%%%%%%%%%%%%%%%%%%%%%%%%%%%%%%%%
%                                   SECTION X                                        %
%%%%%%%%%%%%%%%%%%%%%%%%%%%%%%%%%%%%%%%%%%%%%%%%%%%%%%%%%%%%%%%%%%%%%%%%%%%%%%%%%%%%%%
\section{X}
%%%%%%%%%%%%%%%%%%%%%%%%%%%%%%%%%%%%%%%%%%%%%%%%%%%%%%%%%%%%%%%%%%%%%%%%%%%%%%%%%%%%%%
%                                   SECTION Y                                        %
%%%%%%%%%%%%%%%%%%%%%%%%%%%%%%%%%%%%%%%%%%%%%%%%%%%%%%%%%%%%%%%%%%%%%%%%%%%%%%%%%%%%%%
\section{Y}
%%%%%%%%%%%%%%%%%%%%%%%%%%%%%%%%%%%%%%%%%%%%%%%%%%%%%%%%%%%%%%%%%%%%%%%%%%%%%%%%%%%%%%
%                                   SECTION Z                                        %
%%%%%%%%%%%%%%%%%%%%%%%%%%%%%%%%%%%%%%%%%%%%%%%%%%%%%%%%%%%%%%%%%%%%%%%%%%%%%%%%%%%%%%
\section{Z}
