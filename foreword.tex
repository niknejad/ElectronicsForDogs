\section*{Foreword}
I took this course with Professor Niknejad in the Fall of 2021.  I had transferred from the City College of San Francisco, and just spent my first year at Cal online during the COVID-19 pandemic.  I did not adapt well to the online format, and my first year was rough.  Having gone through the EECS 16A/B series during that year, now EE105 was my first upper division course.  I was very excited for it, but also completely unprepared for the challenge that it would present.

The materials in this course are dense, and they comes at you fast if you don't already have a lot of experience with circuits.  I failed my mid-term exam miserably.  I was a bit shell-shocked from the experience, and I stopped going to class.  I decided to either drop the course, or just fail the rest of it on purpose so I could retake it.

Then one day I got an email from Professor Niknejad.  He was checking in to see if I was alright, and asked about why I hadn't shown up in a few weeks.  This was the first time that a Cal professor had reached out to me--they are usually very busy and hard to get a hold of.  Over that last year I have gotten to know Ali a bit more, and see how busy he really is.  As I write this now, his gesture means even more to me.  I was so stressed out when he contacted me, that I wrote a brutally honest reply.  I told him my plan to drop the class, and went on a rant about the course was not working for me.  I complained that everything went too fast, and this book needed some editing to make it more accessible to people like me.  I had not been to school for fifteen years before returning in 2017, and I always felt like I was playing catch up with all the brilliant students at Cal.

I regretted my email only a few minutes after sending it.  I figured Ali would be offended by it, but to my surprise he wrote a very in-depth reply.  He put himself on my level by telling my how he struggled with the material as a student, and offered words of encouragement to continue; affirming that I would not fail the course.  I decided to stick with the class and work very hard, and I wound up doing much better on the final.

About a month after the class ended, I began revisiting the book.  The material in it is actually really great, and I found it much easier to understand the second time around.  This gave me an idea for a project.  I wanted to edit the book in order to make it more accessible for future students that might have a similar experience in struggling to grasp the material, like I did the first time around.  Fortunately, Professor Niknejad agreed to let me do this.

I hope that the work I put in to this book benefits everyone, but in particular I am hoping to help people out that are in a similar situation that I was.  The bulk of revisions were dedicated to; creating an in-depth index, re-organizing the figures so that they are on the same (or next) page when they are referenced; creating a glossary of terms for a quick reference, and adding in additional mathematical steps to make certain derivations more clear.

Any important terms that are highlighted in \textbf{bold} can be found in the index, and other words that are \textit{italicized} can usually be found in the glossary of terms.  The glossary itself contains both definitions of words and concepts, and their accompanying equations.  The glossary is meant to summarize difficult concepts into a few sentences, using ordinary language.

If you are a bit rusty on circuit analysis, then for a review it is \emph{highly} recommended to read Professor Niknejad's EECS 16A reader.  In particular, you should review all circuit theorems, and make sure that you are very familiar with Thévenin and Norton equivalence.  You can download it at \href{http://rfic.eecs.berkeley.edu/~niknejad/edogs.html}{http://rfic.eecs.berkeley.edu/~niknejad/edogs.html}.

I would like to thank Professor Niknejad for his help in my academic career.  For those taking his class, take advantage of being able to learn circuits from a master of the field.  If you are struggling, remember my story and don't give up.  Never give up.  Good luck with the course, and I hope you find the work put into this material useful in your success.\\[0.5cm]
\noindent
Tarik Fawal\\[0.15cm]
August 18, 2022