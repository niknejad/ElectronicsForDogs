%!TEX root = edance.tex
%%%%%%%%%%%%%%%%
%  APPENDIX A  %
%%%%%%%%%%%%%%%%
\chapter{Appendix: Calculus Review}
%\graphicspath{{./figs_opamp_real/}}
\label{app:calc}
%%%%%%%%%%%%%%%%%%%%%%%%%%%%%%%%%%%%%%%%%%%%%%%%%%%%%%%%%%%%%%%%%%%%%%%%%%%%%%%%%%%%%%%%
%%%%%%%%%%%%%%%%%%%%%%%%%%%%%%%%%%%%%%%%%%%%%%%%%%%%%%%%%%%%%%%%%%%%%%%%%%%%%%%%%%%%%%%%
%                                   SECTION A.1                                        %
%%%%%%%%%%%%%%%%%%%%%%%%%%%%%%%%%%%%%%%%%%%%%%%%%%%%%%%%%%%%%%%%%%%%%%%%%%%%%%%%%%%%%%%%
%%%%%%%%%%%%%%%%%%%%%%%%%%%%%%%%%%%%%%%%%%%%%%%%%%%%%%%%%%%%%%%%%%%%%%%%%%%%%%%%%%%%%%%%
\section{Geometric Series}
\label{sec:geometric}

\section{Taylor Series}
\label{sec:taylor}
Recall from calculus the general form of a power series centered about $a$:
    \begin{equation}
        f(x) = \sum_{n=0}^{\infty} c_n {\big(x - a\big)}^n = c_0 + c_1{\big(x - a\big)}
            + c_2{\big(x - a\big)}^2 + c_3{\big(x - a\big)}^3 + ...
        \label{eq:power_series}
    \end{equation}
How do we solve for the coefficients of this infinite series?

\vspace{0.5cm}
\noindent
We can begin by setting $x = a$ for every term, which yields:
    \begin{equation}
        c_0 = f(a)
        \label{eq:power_first}
    \end{equation}
Now we want to solve for each $n$th consecutive coefficient in the series.  To do this, we take the $n$th derivative of the the series and use the same strategy of setting $x = a$.  This eliminates all terms except for the coefficient of interest we aim to solve for.

\vspace{0.5cm}
\noindent
Starting with the first derivative, we have:
    \begin{equation*}
        \frac{df(x)}{dx} = c_1 + 2c_2{\big(x - a\big)} + 3c_3{\big(x - a\big)}^2
        + 4c_3{\big(x - a\big)}^3 + ...
    \end{equation*}
This yields:
    \begin{equation*}
        c_1 = \frac{1}{1} \cdot \frac{df(a)}{dx}
    \end{equation*}
Continuing with the second derivative:
    \begin{equation*}
        \frac{{df(x)}^2}{d^2x} = (1 \cdot 2)c_2 + (2 \cdot 3)c_3{\big(x - a\big)} + (3 \cdot 4)c_4{\big(x - a\big)}^2
        + ...
    \end{equation*}
Yielding:
    \begin{equation*}
        c_2 = \frac{1}{1 \cdot 2} \cdot \frac{{df(a)}^2}{d^2x}
    \end{equation*}
Then the third derivative:
    \begin{equation*}
        \frac{{df(x)}^3}{d^3x} = (1 \cdot 2 \cdot 3)c_3{\big(x - a\big)}
        + (1 \cdot 2 \cdot 3 \cdot 4)c_4{\big(x - a\big)} + ...
    \end{equation*}
Which yields:
    \begin{equation*}
        c_3 = \frac{1}{1 \cdot 2 \cdot 3} \cdot \frac{{df(a)}^3}{d^3x}
    \end{equation*}
A pattern has presented itself which will continue for all consecutive terms.  From the pattern we can see:
    \begin{equation}
        c_n = \frac{1}{n!} \cdot \frac{{df(a)}^n}{d^nx}
        \label{eq:power_co}
    \end{equation}
Plugging \emph{Eq.~\ref{eq:power_first}} and \emph{Eq.~\ref{eq:power_co}} into \emph{Eq.~\ref{eq:power_series}} gives us the general form of a Taylor series using prime notation:
    \begin{equation}
        f(x) = \sum_{n=0}^{\infty}\frac{f^{(n)}(a)}{n!}{\big(x - a\big)}^n = f(a)
        + \frac{f'(a)}{1!}{\big(x - a\big)} + \frac{f''(a)}{2!}{\big(x - a\big)}^2 + ...
        \label{eq:taylor_series}
    \end{equation}
For the special case where $a = 0$ in \emph{Eq.~\ref{eq:taylor_series}}, we have the general form of a Maclaurin series:
    \begin{equation}
        f(x) = \sum_{n=0}^{\infty}\frac{f^{(n)}(0)}{n!}{x}^n = f(0)
        + \frac{f'(0)}{1!}{x} + \frac{f''(0)}{2!}{x}^2 + ...
        \label{eq:maclaurin_series}
    \end{equation}
