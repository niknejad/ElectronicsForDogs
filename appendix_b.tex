%!TEX root = edance.tex
%%%%%%%%%%%%%%%%
%  APPENDIX B  %
%%%%%%%%%%%%%%%%
\chapter{Appendix: Linear Algebra Review}
%\graphicspath{{./figs_opamp_real/}}
%%%%%%%%%%%%%%%%%%%%%%%%%%%%%%%%%%%%%%%%%%%%%%%%%%%%%%%%%%%%%%%%%%%%%%%%%%%%%%%%%%%%%%%%
%%%%%%%%%%%%%%%%%%%%%%%%%%%%%%%%%%%%%%%%%%%%%%%%%%%%%%%%%%%%%%%%%%%%%%%%%%%%%%%%%%%%%%%%
%                                   SECTION B.1                                        %
%%%%%%%%%%%%%%%%%%%%%%%%%%%%%%%%%%%%%%%%%%%%%%%%%%%%%%%%%%%%%%%%%%%%%%%%%%%%%%%%%%%%%%%%
%%%%%%%%%%%%%%%%%%%%%%%%%%%%%%%%%%%%%%%%%%%%%%%%%%%%%%%%%%%%%%%%%%%%%%%%%%%%%%%%%%%%%%%%
\section{Linear Dependence and Independence}
null
\subsection{Proof of Sine and Cosine's Linear Independence}
\textbf{Claim. } $a \cdot \sin(x) + b \cdot \cos(x) = 0$ is linearly independent.
    \begin{proof}
        Using the definition of linear independence, we will show that $a$ and $b$ must be zero for all $x$ in some interval.  Since sine and cosine are periodic, we will show that $a$ and $b$ must be $0$ in the interval $[0, 2\pi]$.  Then it follows that $a$ and $b$ must be $0$ for all $x$ in $[-\infty, \infty]$.
        
        \vspace{0.5cm}
        \noindent
        We can find $a$ by setting $x = \frac{\pi}{2}$:
        \begin{align*}
            a \cdot \cancelto{1}{\sin(\frac{\pi}{2})} + b \cdot \cancelto{0}{\cos(\frac{\pi}{2})} &= 0\\
            a &= 0
        \end{align*}
        Next, we can find $b$ by setting $x = 0$:
        \begin{align*}
            a \cdot \cancelto{0}{\sin(0)} + b \cdot \cancelto{1}{\cos(0)} &= 0\\
            b &= 0
        \end{align*}
    \end{proof}
